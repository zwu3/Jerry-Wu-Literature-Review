\documentclass[10pt,twocolumn]{article} 

\usepackage{oxycomps} % use the main oxycomps style file

\bibliography{references}

\pdfinfo{
    /Title (DeFi: Blockchain Time Locked Wallet)
    /Author (Jerry Wu)
}

\title{DeFi: Blockchain Time Locked Wallet}

\author{Jerry Wu}
\affiliation{Occidental College}
\email{zwu@oxy.edu}

\begin{document}

\maketitle

\begin{abstract}
    This document serves as an introduction to my COMPS on the topic of Decentralized Finance. I will using my project "Blockchain Time Locked Wallet" to demonstrate
    some of my arguements and evaulations.
\end{abstract}

\section{Problem Context}

With Decentralized Finance emerging in the financial technology scene, it is easy to oversee the potential of the new way of currency trading similar to those used by cryptocurrencies. Money is usually kept by banks, corporations whose overarching objective is to make money through centralized finance. Third parties who facilitate money flow between parties abound in the financial system, each charging a charge for their services. By allowing people to perform financial transactions using developing technologies, decentralized finance eliminates intermediaries. Peer-to-peer financial networks that use security protocols, connectivity, software, and hardware developments are used to achieve this. The purpose of this project is to have an investigation on the doability of this system in daily life setting targeting student groups, with the use of digital wallet.
The evolution of decentralized finance is still in its early phases. For starters, it is unregulated, which means that infrastructure failures, hacks, and frauds continue to plague the ecosystem. Current laws are based on the concept of distinct financial jurisdictions, each with its own set of laws and regulations. The potential of DeFi to conduct borderless transactions raises important problems for this form of regulation.

\section{Technical Background}
Blockchain technology, which is also used in cryptocurrencies, is used in decentralized finance. A distributed and secure database or ledger is referred to as a blockchain. dApps are the applications that conduct transactions and run the blockchain. Transactions are stored in blocks on the blockchain and subsequently validated by other users. If all of the verifiers agree on a transaction, the block is closed and encrypted, and a new block is created containing information from the preceding block. The information in each subsequent block "chains" the blocks together, giving the blockchain its name. There is no method to edit a blockchain since information in prior blocks cannot be modified without impacting subsequent blocks. The secure nature of a blockchain is provided by this notion, as well as other security mechanisms.
	The blockchain's novelty is that it ensures the fidelity and security of a data record while also generating trust without the requirement for a trusted third party.
	For the project, it will be built on the Ethereum Blockchain, which is a decentralized, open-source blockchain with smart contract functionality. With the use of Truffle framework, the provided resources within the framework should make the time-locked wallet doable.

\section{Prior Work}
There has been research on the growth of DeFi assets. A research done by Fabian Schar, “Decentralized Finance: On Blockchain - and Smart Contract-Based Financial Markets' ', suggests that Defi may become relevant in a much broader context and has sparked interest among policymakers, researchers, and financial institutions. Compared to this research, it will target more on individuals that are still in college.
More research is needed.

\section{The Oxy CS Comps Paper}

\subsection{Goals}

\subsection{Audience}

\subsection{Requirements}

\section{Sections of the Oxy CS Comps Paper}

\subsection{Introduction and Background}

\subsection{Prior Work}

\subsection{Methods}

\subsection{Evaluation}

\subsection{Ethical Considerations}

\subsection{Limitations, Future Work, and Conclusion}

\subsection{Appendices}

\section{Conclusion}

\printbibliography 

\end{document}
